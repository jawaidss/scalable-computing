%%%
% CSSE lab template
% Written March 28, 2011 by Tim Ekl
%
% This template is part of a template system; you should be generating a lab
% using the newlab.sh file. Once you have generated the lab, feel free to
% remove this comment.
%
% This lab has five TODO items that need to be completed manually after
% generating the lab from template. Suggestions are provided with each TODO.
%
% For more info and examples of labs built from this template, see:
% http://github.com/CSSE376/Labs (note you must be a 376 TA to access)
%%%

\documentclass{article}

\newcommand{\labduedate}{at the end of Week 6}

% Actual required things start here
\usepackage{array}
\usepackage{mdwlist}
\usepackage{fancyhdr}
\usepackage[usenames,dvipsnames]{color}
\usepackage{graphicx}
\usepackage{minted}
\usepackage{fullpage}
\usepackage[colorlinks=true]{hyperref}
\usepackage{textcomp}
\usepackage{totcount}

% Helpful shortcuts for the document itself
\newcommand{\productname}{Azure}
\newcommand{\longproductname}{Windows Azure}
\newcommand{\labnumber}{7}


\newtotcounter{questions}\setcounter{questions}{0}
\newcommand{\question}[1]{
\addtocounter{questions}{1}
\ifnum1=#1
\textbf{[Question \arabic{questions} \textrm{\textit{(#1 pt)}}] }
\else
\textbf{[Question \arabic{questions} \textrm{\textit{(#1 pts)}}] }
\fi
}

\newtotcounter{tasks}\setcounter{tasks}{0}
\newcommand{\task}[2]{
\addtocounter{tasks}{1}
\setcounter{subtasks}{0}
\ifnum0=#1
\subsection*{Task \arabic{tasks}: #2}
\else
\subsection*{Task \arabic{tasks}: #2 \textit{(#1 pts)}}
\fi
}

\newtotcounter{subtasks}\setcounter{subtasks}{0}
\newcommand{\subtask}[2]{
\addtocounter{subtasks}{1}
\ifnum0=#1
\subsubsection*{Subtask \arabic{tasks}.\arabic{subtasks}: #2}
\else
\subsubsection*{Task \arabic{tasks}.\arabic{subtasks}: #2 \textit{(#1 pts)}}
\fi
}

\pagestyle{fancy}
\headheight 24pt
\begin{document}

\chead{\textcolor{Gray}{CSSE491 -- Scalable Computing Lab Assignment}}
\headsep = 24pt

\begin{center}
{ \large
\textbf{Lab \labnumber: \longproductname} \\
\textbf{Amazon EC2}
}
\end{center}

\subsection*{Objective}
% TODO: What is the objective of this lab? What should students be learning?

\subsection*{Lab format}
% TODO: What format is the lab in? Is it entirely written? Code-based? Are there auxiliary documents that supplement the lab (e.g. videos, code packages)?

\subsection*{Required materials}
% TODO: What do students need to complete the lab? (Common items include a laptop and an Internet connection.)

\subsection*{Grading rubric}
\begin{tabular}{p{5.5in} r}
  % TODO: Where are points coming from in this lab? (Usually this is the last section filled in, as it must be listed and totalled manually.)
\end{tabular}

\subsection*{Due date}
The lab is due \textcolor{red}{\textbf{\labduedate}}.

\subsection*{Turn-in instructions}
Create a personal Git repository on GitHub named ``Lab\labnumber-username'', where \verb!username! is your Kerberos username, and commit the following files to it:
\begin{itemize}
\item Edited project files from \productname (if any)
\item A PDF file with the answers to each question \textit{clearly marked}. When you have completed this lab, you will have answered \total{questions} questions.
\item A \verb!who.txt! file with your name. If you worked with someone else, include their name in the \verb!who.txt! file.
\end{itemize}

Once committed, push all files to GitHub. You may work with a partner, as long as each of you pushes your own copy of all of the listed files. Note that files submitted via ANGEL, email, or other methods will receive zero credit. In addition, files that cannot be accessed without additional work (e.g. zip files, LaTeX documents that require compiling) will lose you points. After you push your files to GitHub, please complete the feedback survey on Angel Under Labs - Labs Anonymous Feedback!


\task{5}{Sign Up for AWS}

Before we get started, you'll need a free account with AWS. Do the following:

\begin{enumerate*}
\item http://aws.amazon.com/free/
\item Click `Sign Up Now'
\item Log in with an Amazon account, or put in your email and choose `I am a new user.'
\item Follow the wizard to create your account
\end{enumerate*}

\task{5}{Creating a new Elastic Compute Cloud (EC2) instance}

Start at the URL \href{https://console.aws.amazon.com}{https://console.aws.amazon.com}. Do the following:

\begin{enumerate*}
\item Click the `Amazon EC2' tab
\item Click `Instances' on the left sidebar
\item Click `Launch Instance' on the top toolbar
\item Use the Quick Launch Wizard
\end{enumerate*}

\subsubsection*{Screen 1}

\includegraphics[width=6in]{screen1}

\begin{enumerate*}
\item Name Your Instance: Be creative. Or don't. But name it something
\item Choose a Key Pair: Use an existing pair if you've created one. Otherwise use Create New with Name: your username. Let the key download and remember where it is.
\item Choose a Launch Configuration: Use a modern version of 64-bit Ubuntu Server, preferably one that matches your local configuration.
\item Click `Continue'.
\end{enumerate*}

\subsubsection*{Screen 2}

\includegraphics[width=6in]{screen2}

Simply click `Edit details'

\subsubsection*{Screen 3}

\includegraphics[width=6in]{screen3}

\begin{enumerate*}
\item Instance Details
\begin{enumerate*}
\item Type: t1.micro. Unless you want to actually owe Amazon money
\end{enumerate*}
\item Security Settings - Click `Create new Security Group'
\begin{enumerate*}
\item Group name/Description: Your call.
\item New rules
\begin{enumerate*}
\item Create a new rule: SSH, HTTP, Custom TCP Rule
\item Port range: n/a, n/a, 3000
\item Source: 0.0.0.0/0
\end{enumerate*}
\item Click `Create'
\end{enumerate*}
\item Click `Save details'
\end{enumerate*}

\subsubsection*{Screen 4}

Click `Launch'

\task{5}{Pairing the Instance}

In this section, we discuss pairing your new EC2 instance with an Elastic IP. Again starting from \href{https://console.aws.amazon.com}{https://console.aws.amazon.com}, do the following:

\begin{enumerate*}
\item Click the `Amazon EC2' tab
\item Click `Elastic IPs' on the left sidebar
\item Click `Allocate New Address' on the top toolbar
\item Click `Yes, Allocate'
\item Check the box of your new Elastic IP and click `Associate Address'
\begin{enumerate*}
\item Instance: Select your new EC2 instance.
\item Click `Yes, Associate'
\end{enumerate*}
\end{enumerate*}

\task{5}{Connecting to the Instance}

Fire up a shell with SSH, and run:

\begin{verbatim}
$ ssh -i <key pair name>.pem ubuntu@<your elastic IP>
\end{verbatim}

\subsection*{Turn-in instructions}
Create a personal Git repository on GitHub named ``Lab\labnumber-username'', where \verb!username! is your Kerberos username, and commit the following files to it:
\begin{itemize}
\item Edited project files from \productname (if any)
\item A PDF file with the answers to each question \textit{clearly marked}. When you have completed this lab, you will have answered \total{questions} questions.
\item A \verb!who.txt! file with your name. If you worked with someone else, include their name in the \verb!who.txt! file.
\end{itemize}

Once committed, push all files to GitHub. You may work with a partner, as long as each of you pushes your own copy of all of the listed files. Note that files submitted via ANGEL, email, or other methods will receive zero credit. In addition, files that cannot be accessed without additional work (e.g. zip files, LaTeX documents that require compiling) will lose you points. After you push your files to GitHub, please complete the feedback survey on Angel Under Labs - Labs Anonymous Feedback!


\subsection*{Revision History}
\begin{itemize*}
 \item Fri Apr 20 10:29:39 EDT 2012: Lab created with template generator
\end{itemize*}

\end{document}
