%%%
% CSSE lab template
% Written March 28, 2011 by Tim Ekl
%
% This template is part of a template system; you should be generating a lab
% using the newlab.sh file. Once you have generated the lab, feel free to
% remove this comment.
%
% This lab has five TODO items that need to be completed manually after
% generating the lab from template. Suggestions are provided with each TODO.
%
% For more info and examples of labs built from this template, see:
% http://github.com/CSSE376/Labs (note you must be a 376 TA to access)
%%%

\documentclass{article}

\newcommand{\labduedate}{at the end of week 5}

% Actual required things start here
\usepackage{array}
\usepackage{mdwlist}
\usepackage{fancyhdr}
\usepackage[usenames,dvipsnames]{color}
\usepackage{graphicx}
\usepackage{minted}
\usepackage{fullpage}
\usepackage[colorlinks=true]{hyperref}
\usepackage{textcomp}
\usepackage{totcount}

% Helpful shortcuts for the document itself
\newcommand{\productname}{Azure}
\newcommand{\longproductname}{Windows Azure}
\newcommand{\labnumber}{7}


\newtotcounter{questions}\setcounter{questions}{0}
\newcommand{\question}[1]{
\addtocounter{questions}{1}
\ifnum1=#1
\textbf{[Question \arabic{questions} \textrm{\textit{(#1 pt)}}] }
\else
\textbf{[Question \arabic{questions} \textrm{\textit{(#1 pts)}}] }
\fi
}

\newtotcounter{tasks}\setcounter{tasks}{0}
\newcommand{\task}[2]{
\addtocounter{tasks}{1}
\setcounter{subtasks}{0}
\ifnum0=#1
\subsection*{Task \arabic{tasks}: #2}
\else
\subsection*{Task \arabic{tasks}: #2 \textit{(#1 pts)}}
\fi
}

\newtotcounter{subtasks}\setcounter{subtasks}{0}
\newcommand{\subtask}[2]{
\addtocounter{subtasks}{1}
\ifnum0=#1
\subsubsection*{Subtask \arabic{tasks}.\arabic{subtasks}: #2}
\else
\subsubsection*{Task \arabic{tasks}.\arabic{subtasks}: #2 \textit{(#1 pts)}}
\fi
}

\pagestyle{fancy}
\headheight 24pt
\begin{document}

\chead{\textcolor{Gray}{CSSE491 -- Scalable Computing Lab Assignment}}
\headsep = 24pt

\begin{center}
{ \large
\textbf{Lab \labnumber: \longproductname} \\
\textbf{Tags}
}
\end{center}

\subsection*{Objective}
The objective of this lab is to expose students to one of the most important facets of scalable computing: search. By adding tags to photos stored in the HelloWorld app from Lab 4, this lab aims to explore search implementations in Node.js across large databases -- a key concept in scalable systems.

\subsection*{Lab format}
This lab, like most previous labs, is largely guided. However, it will require students to answer a number of comprehension questions, some of which may require independent research. There is also a small independent coding portion near the end of the lab.

\subsection*{Required materials}
Students will need:
\begin{itemize*}
\item The HelloWorld app from Lab 4
\item A laptop with an Internet connection
\item Photos from \href{http://interfacelift.com}{interfacelift} in \verb!public/img/interfacelift!
\end{itemize*}

\subsection*{Grading rubric}
\begin{tabular}{p{5.5in} r}
Tasks 1 and 3 are worth \textbf{5} points each & $2 \times 5 = 10$ \\
Task 2 is worth \textbf{10} points & $1 \times 10 = 10$ \\
Questions 1-3 are worth \textbf{5} points each & $3 \times 5 = 15$ \\ \hline
& \textbf{35} points
\end{tabular}

\subsection*{Due date}
The lab is due \textcolor{red}{\textbf{\labduedate}}.

\subsection*{Turn-in instructions}
Create a personal Git repository on GitHub named ``Lab\labnumber-username'', where \verb!username! is your Kerberos username, and commit the following files to it:
\begin{itemize}
\item Edited project files from \productname (if any)
\item A PDF file with the answers to each question \textit{clearly marked}. When you have completed this lab, you will have answered \total{questions} questions.
\item A \verb!who.txt! file with your name. If you worked with someone else, include their name in the \verb!who.txt! file.
\end{itemize}

Once committed, push all files to GitHub. You may work with a partner, as long as each of you pushes your own copy of all of the listed files. Note that files submitted via ANGEL, email, or other methods will receive zero credit. In addition, files that cannot be accessed without additional work (e.g. zip files, LaTeX documents that require compiling) will lose you points. After you push your files to GitHub, please complete the feedback survey on Angel Under Labs - Labs Anonymous Feedback!


\task{5}{Update the Model}

Open the file \verb!models.js!, and make the following changes:

\begin{itemize*}
\item Add an attribute called \verb!tags! to the PhotoSchema definition of type String
\item In the function beginning \verb!Photo.count(function(...!, add the following:
\begin{itemize*}
\item A list of tags after the definition \verb!var FOLDER...!. This list should take the form:

\begin{minted}{javascript}
var TAGS = ['foo', 'bar', 'your', 'tags', 'here'];
\end{minted}

\item The following code block in the set of lines containing \verb!photo.path = ...!:

\begin{minted}{javascript}
var j = 0;
var n = Math.ceil(Math.random() * 5);
while (j < n) {
  var tag = TAGS[Math.round(Math.random() * (TAGS.length - 1))];
  if (photo.tags.indexOf(tag) == -1) {
    photo.tags.push(tag);
  }
  j++;
}
\end{minted}
\end{itemize*}
\end{itemize*}

\question{5} Explain, in your own words, what the new code block in the last bullet point above does.

\task{10}{Routing}

In the file \verb!app.js!, change:

\begin{minted}{javascript}
app.get('/photos/:page', routes.photos.all);
\end{minted}

to

\begin{minted}{javascript}
app.get('/photos/:page/:tag?', routes.photos.all);
\end{minted}

\question{5} We know that the colon indicates a route parameter; what do you think the question mark (e.g. \verb!:tag?!) indicates? Feel free to refer to Node.js or Express documentation.

Once the app is updated, edit the route it refers to. Open the file \verb!routes/photos.js! and make the following changes:

\begin{itemize}
\item Update \verb!exports.all = function(...! (after \verb!var page = ...;!) with:
\begin{minted}{javascript}
  var tag = req.params.tag;
  var query = {};
  if (tag) {
    query = {tags: {$in: [tag]}};
  }
\end{minted}

\item Update the same function (after \verb!if (page == 0) {...}!) with:

\begin{minted}{javascript}
  options.sort = {description: 1};
\end{minted}

\item Update the same function (after \verb!context.title = ...;!) with:

\begin{minted}{javascript}
    context.tag = tag;
    if (tag) {
      context.title += ' - ' + tag;
    }
\end{minted}

\item Finally, change
\begin{minted}{javascript}
Photo.find({}, ['path', 'description'], ...);
\end{minted}
to
\begin{minted}{javascript}
Photo.find(query, ['path', 'description', 'tags'], ...);
\end{minted}
\end{itemize}

\question{5} Describe, in your own words, what the updates listed above do.

\task{5}{Update the View}

So what's the desired user interface for tags? They should, at a minimum:

\begin{itemize*}
\item Be displayed below their respective photos
\item Link to result sets for the same tag
\end{itemize*}

To accomplish this, open \verb!views/photos.jade!. Add (after both \verb!href='/photos/' + (page + 1)! and \verb!href='/photos/' + (page - 1)!):

\begin{minted}{jade}
 + (tag && ('/' + tag) || '')
\end{minted}

Also, add (before \verb!div.alert.alert-info="..."!):

\begin{minted}{jade}
  div.btn-group
    each tag in center.tags
      a.btn(href='/photos/all/' + tag)=tag
  br
\end{minted}

\subsection*{Run the App}

Like always, test with:

\begin{itemize*}
\item \verb!node app.js!
\item \href{http://localhost:3000/photos}{http://localhost:3000/photos}
\end{itemize*}

\subsection*{Turn-in instructions}
Create a personal Git repository on GitHub named ``Lab\labnumber-username'', where \verb!username! is your Kerberos username, and commit the following files to it:
\begin{itemize}
\item Edited project files from \productname (if any)
\item A PDF file with the answers to each question \textit{clearly marked}. When you have completed this lab, you will have answered \total{questions} questions.
\item A \verb!who.txt! file with your name. If you worked with someone else, include their name in the \verb!who.txt! file.
\end{itemize}

Once committed, push all files to GitHub. You may work with a partner, as long as each of you pushes your own copy of all of the listed files. Note that files submitted via ANGEL, email, or other methods will receive zero credit. In addition, files that cannot be accessed without additional work (e.g. zip files, LaTeX documents that require compiling) will lose you points. After you push your files to GitHub, please complete the feedback survey on Angel Under Labs - Labs Anonymous Feedback!


\subsection*{Revision History}
\begin{itemize*}
 \item Sat Apr 14 16:13:11 EDT 2012: Lab created with template generator
\end{itemize*}

\end{document}
