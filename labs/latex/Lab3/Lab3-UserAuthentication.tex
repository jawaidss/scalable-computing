%%%
% CSSE lab template
% Written March 28, 2011 by Tim Ekl
%
% This template is part of a template system; you should be generating a lab
% using the newlab.sh file. Once you have generated the lab, feel free to
% remove this comment.
%
% This lab has five TODO items that need to be completed manually after
% generating the lab from template. Suggestions are provided with each TODO.
%
% For more info and examples of labs built from this template, see:
% http://github.com/CSSE376/Labs (note you must be a 376 TA to access)
%%%

\documentclass{article}

\newcommand{\labduedate}{at the end of Week 3}

% Actual required things start here
\usepackage{array}
\usepackage{mdwlist}
\usepackage{fancyhdr}
\usepackage[usenames,dvipsnames]{color}
\usepackage{graphicx}
\usepackage{minted}
\usepackage{fullpage}
\usepackage[colorlinks=true]{hyperref}
\usepackage{textcomp}
\usepackage{totcount}

% Helpful shortcuts for the document itself
\newcommand{\productname}{Node.js}
\newcommand{\longproductname}{Node.js (with Express and Mongoose)}
\newcommand{\labnumber}{4}


\newtotcounter{questions}\setcounter{questions}{0}
\newcommand{\question}[1]{
\addtocounter{questions}{1}
\ifnum1=#1
\textbf{[Question \arabic{questions} \textrm{\textit{(#1 pt)}}] }
\else
\textbf{[Question \arabic{questions} \textrm{\textit{(#1 pts)}}] }
\fi
}

\newtotcounter{tasks}\setcounter{tasks}{0}
\newcommand{\task}[2]{
\addtocounter{tasks}{1}
\setcounter{subtasks}{0}
\ifnum0=#1
\subsection*{Task \arabic{tasks}: #2}
\else
\subsection*{Task \arabic{tasks}: #2 \textit{(#1 pts)}}
\fi
}

\newtotcounter{subtasks}\setcounter{subtasks}{0}
\newcommand{\subtask}[2]{
\addtocounter{subtasks}{1}
\ifnum0=#1
\subsubsection*{Subtask \arabic{tasks}.\arabic{subtasks}: #2}
\else
\subsubsection*{Task \arabic{tasks}.\arabic{subtasks}: #2 \textit{(#1 pts)}}
\fi
}

\pagestyle{fancy}
\headheight 24pt
\begin{document}

\chead{\textcolor{Gray}{CSSE491 -- Scalable Computing Lab Assignment}}
\headsep = 24pt

\begin{center}
{ \large
\textbf{Lab \labnumber: \longproductname} \\
\textbf{User Authentication}
}
\end{center}

\subsection*{Objective}
Up to this point, we've been writing a public webapp: anyone with the URL can view or modify data. However, security is a huge (and growing) concern online; this lab aims to introduce security concepts by integrating user authentication (via Twitter) with the existing HelloWorld app from labs 1 and 2.

\subsection*{Lab format}
As with previous labs, this lab is mostly guided. Students will get authentication tokens from Twitter, then modify existing app files to work against the Twitter user framework.

\subsection*{Required materials}
Students will need a laptop, an Internet connection, a Twitter account, and the HelloWorld app from labs 1 and 2.

\subsection*{Grading rubric}
\begin{tabular}{p{5.5in} r}
Question 1 is worth \textbf{5} points & $1 \times 5 = 5$ \\
Task 1 is worth \textbf{5} points & $1 \times 5 = 5$ \\
Tasks 2-4 are worth \textbf{10} points each & $3 \times 10 = 30$ \\ \hline
& \textbf{40} points
\end{tabular}

\subsection*{Due date}
The lab is due \textcolor{red}{\textbf{\labduedate}}.

\subsection*{Turn-in instructions}
Create a personal Git repository on GitHub named ``Lab\labnumber-username'', where \verb!username! is your Kerberos username, and commit the following files to it:
\begin{itemize}
\item Edited project files from \productname
\item A PDF file with the answers to each question \textit{clearly marked}. When you have completed this lab, you will have answered \total{questions} questions.
\item A \verb!who.txt! file with your name. If you worked with someone else, include their name in the \verb!who.txt! file.
\end{itemize}

Once committed, push all files to GitHub. You may work with a partner, as long as each of you pushes your own copy of all of the listed files. Note that files submitted via ANGEL, email, or other methods will receive zero credit. In addition, files that cannot be accessed without additional work (e.g. zip files, LaTeX documents that require compiling) will lose you points. After you push your files to GitHub, please complete the feedback survey on Angel Under Labs - Labs Anonymous Feedback!


\task{5}{Twitter Keys}

Before we begin development, you'll need to fetch some authentication info from Twitter. If you haven't already, \href{http://twitter.com}{sign up} for a Twitter account. Then, at a shell in your HelloWorld directory, type:

\begin{minted}{bash}
$ sudo apt-get install nodejs-dev
$ npm install mongoose-auth
\end{minted}

You'll also need to alias the hostname \verb!local.host! to localhost; edit the file \verb!/etc/hosts! in your preferred text editor, and make sure it contains the line:

\begin{verbatim}
127.0.0.1 localhost local.host
\end{verbatim}

Now go to your \href{https://dev.twitter.com/apps}{Twitter app dev page} and add the following information:

\begin{center}
\begin{tabular}{p{1in} | p{4in}} \hline
\textbf{Access} & Read-only \\ \hline
\textbf{Callback URL} & \verb!http://local.host:3000/auth/twitter/callback! \\ \hline
\end{tabular}
\end{center}

Finally, create a new file called \verb!settings.js! in the root of your HelloWorld directory and add the following lines, replacing as appropriate:

\begin{minted}{javascript}
exports.SECRET_KEY = 'YOUR SECRET KEY';
exports.TWITTER_CONSUMER_KEY = 'YOUR CONSUMER KEY';
exports.TWITTER_CONSUMER_SECRET = 'YOUR CONSUMER SECRET';
\end{minted}

\task{10}{Modifying the App}

Open the file \verb!app.js!. Add near the top:

\begin{minted}{javascript}
var mongooseAuth = require('mongoose-auth');

var SECRET_KEY = require('./settings.js').SECRET_KEY;
\end{minted}

Now we need to remove some standard middleware and replace it with others. Comment the line:

\begin{minted}{javascript}
app.use(app.router);
\end{minted}

After this newly commented line, add:

\begin{minted}{javascript}
app.use(express.cookieParser());
app.use(express.session({secret: SECRET_KEY}));
app.use(mongooseAuth.middleware());
\end{minted}

Finally, add:

\begin{minted}{javascript}
mongooseAuth.helpExpress(app);
\end{minted}

\question{5} Why do you think we needed to remove the first middleware item (\verb!app.router!) before adding the others?

\task{10}{Updating the Models}

Our app file now supports authentication with the mongoose-auth package; however, we have yet to define what our user schema looks like in a way that our app can understand. Open the file \verb!models.js! and add the following block:

\begin{minted}{javascript}
var mongooseAuth = require('mongoose-auth');

var TWITTER_CONSUMER_KEY = require('./settings.js').TWITTER_CONSUMER_KEY;
var TWITTER_CONSUMER_SECRET = require('./settings.js').TWITTER_CONSUMER_SECRET;

var UserSchema = new mongoose.Schema();
UserSchema.plugin(mongooseAuth, {
    everymodule: {
        everyauth: {
            User: function() {
                return User;
            }   
        }   
    },  
    twitter: {
        everyauth: {
            myHostname: 'http://local.host:3000',
            consumerKey: TWITTER_CONSUMER_KEY,
            consumerSecret: TWITTER_CONSUMER_SECRET,
            redirectPath: '/' 
        }   
    }
});

var User = mongoose.model('User', UserSchema);
\end{minted}

\task{10}{Showing Login Status}

As with in lab 2, we've done everything on the backend to support a new feature, but where does it appear? We need to edit the view again. Open the file \verb!views/index.jade! and add:

\begin{minted}{jade}
h2 Authentication
- if(!everyauth.loggedIn)
  p
    a(href="/auth/twitter") Log in with Twitter
- else
  p #{everyauth.twitter.user.name}
  p #{everyauth.twitter.user.location}
  p
    a(href="http://www.twitter.com/" + everyauth.twitter.user.screen_name) @#{everyauth.twitter.user.screen_name}
  p
    a(href="/logout") Log out
\end{minted}

\subsection*{Try It Out}

As always, go to the URL \href{http://localhost:3000/}{http://localhost:3000} to try your new feature.

\subsection*{Turn-in instructions}
Create a personal Git repository on GitHub named ``Lab\labnumber-username'', where \verb!username! is your Kerberos username, and commit the following files to it:
\begin{itemize}
\item Edited project files from \productname
\item A PDF file with the answers to each question \textit{clearly marked}. When you have completed this lab, you will have answered \total{questions} questions.
\item A \verb!who.txt! file with your name. If you worked with someone else, include their name in the \verb!who.txt! file.
\end{itemize}

Once committed, push all files to GitHub. You may work with a partner, as long as each of you pushes your own copy of all of the listed files. Note that files submitted via ANGEL, email, or other methods will receive zero credit. In addition, files that cannot be accessed without additional work (e.g. zip files, LaTeX documents that require compiling) will lose you points. After you push your files to GitHub, please complete the feedback survey on Angel Under Labs - Labs Anonymous Feedback!


\subsection*{Revision History}
\begin{itemize*}
\item April 13, 2012: Lab converted to LaTeX by Tim Ekl
\item March 13, 2012: Lab written by Samad Jawaid
\end{itemize*}

\end{document}
