%%%
% CSSE lab template
% Written March 28, 2011 by Tim Ekl
%
% This template is part of a template system; you should be generating a lab
% using the newlab.sh file. Once you have generated the lab, feel free to
% remove this comment.
%
% This lab has five TODO items that need to be completed manually after
% generating the lab from template. Suggestions are provided with each TODO.
%
% For more info and examples of labs built from this template, see:
% http://github.com/CSSE376/Labs (note you must be a 376 TA to access)
%%%

\documentclass{article}

\newcommand{\labduedate}{at the end of Week 1}

% Actual required things start here
\usepackage{array}
\usepackage{mdwlist}
\usepackage{fancyhdr}
\usepackage[usenames,dvipsnames]{color}
\usepackage{graphicx}
\usepackage{minted}
\usepackage{fullpage}
\usepackage[colorlinks=true]{hyperref}
\usepackage{textcomp}
\usepackage{totcount}

% Helpful shortcuts for the document itself
\newcommand{\productname}{Azure}
\newcommand{\longproductname}{Windows Azure}
\newcommand{\labnumber}{7}


\newtotcounter{questions}\setcounter{questions}{0}
\newcommand{\question}[1]{
\addtocounter{questions}{1}
\ifnum1=#1
\textbf{[Question \arabic{questions} \textrm{\textit{(#1 pt)}}] }
\else
\textbf{[Question \arabic{questions} \textrm{\textit{(#1 pts)}}] }
\fi
}

\newtotcounter{tasks}\setcounter{tasks}{0}
\newcommand{\task}[2]{
\addtocounter{tasks}{1}
\setcounter{subtasks}{0}
\ifnum0=#1
\subsection*{Task \arabic{tasks}: #2}
\else
\subsection*{Task \arabic{tasks}: #2 \textit{(#1 pts)}}
\fi
}

\newtotcounter{subtasks}\setcounter{subtasks}{0}
\newcommand{\subtask}[2]{
\addtocounter{subtasks}{1}
\ifnum0=#1
\subsubsection*{Subtask \arabic{tasks}.\arabic{subtasks}: #2}
\else
\subsubsection*{Task \arabic{tasks}.\arabic{subtasks}: #2 \textit{(#1 pts)}}
\fi
}

\pagestyle{fancy}
\headheight 24pt
\begin{document}

\chead{\textcolor{Gray}{CSSE491 -- Scalable Computing Lab Assignment}}
\headsep = 24pt

\begin{center}
{ \large
\textbf{Lab \labnumber: \longproductname} \\
\textbf{Getting Started}
}
\end{center}

\subsection*{Objective}
The goal of this lab is to introduce students to server-side webapp programming using a number of relatively new frameworks. By the end of the lab, students should be acquainted with Node.js, MongoDB (and its Node library Mongoose), and Express.

\subsection*{Lab format}
As an introductory assignment, this lab primarily guides the user through the installation of Node.js components and dependencies, then instructs students on construction of a basic HelloWorld app. The lab is primarily guided.

\subsection*{Required materials}
Students will need a laptop with an active Internet connection.

\subsection*{Grading rubric}
\begin{tabular}{p{5.5in} r}
Questions 1-3 are worth \textbf{5} points each. & $3 \times 5 = 15$ \\
Tasks 1-4 are worth \textbf{5} points each. & $4 \times 5 = 20$ \\
Tasks 5-7 are worth \textbf{10} points each. & $3 \times 10 = 30$ \\ \hline
& \textbf{65} points
\end{tabular}

\subsection*{Due date}
The lab is due \textcolor{red}{\textbf{\labduedate}}.

\subsection*{Turn-in instructions}
Create a personal Git repository on GitHub named ``Lab\labnumber-username'', where \verb!username! is your Kerberos username, and commit the following files to it:
\begin{itemize}
\item Edited project files from \productname (if any)
\item A PDF file with the answers to each question \textit{clearly marked}. When you have completed this lab, you will have answered \total{questions} questions.
\item A \verb!who.txt! file with your name. If you worked with someone else, include their name in the \verb!who.txt! file.
\end{itemize}

Once committed, push all files to GitHub. You may work with a partner, as long as each of you pushes your own copy of all of the listed files. Note that files submitted via ANGEL, email, or other methods will receive zero credit. In addition, files that cannot be accessed without additional work (e.g. zip files, LaTeX documents that require compiling) will lose you points. After you push your files to GitHub, please complete the feedback survey on Angel Under Labs - Labs Anonymous Feedback!


\task{5}{Install Node.js}

\begin{minted}{bash}
sudo apt-get install build-essential
sudo apt-get install python-software-properties
sudo add-apt-repository ppa:chris-lea/node.js
sudo apt-get update
sudo apt-get install nodejs
\end{minted}

\textit{Note: without the first three commands, a very old version may be installed.}

\task{5}{Install MongoDB}

\begin{minted}{bash}
sudo apt-key adv --keyserver keyserver.ubuntu.com --recv 7F0CEB10
echo "deb http://downloads-distro.mongodb.org/repo/ubuntu-upstart dist 10gen" | sudo tee -a /etc/apt/sources.list
sudo apt-get update
sudo apt-get install mongodb-10gen
\end{minted}

\textit{Note: With just the package} \verb!mongodb!, \textit{an old version may be installed.}

\task{5}{Install NPM}

\begin{minted}{bash}
sudo apt-get install curl
curl http://npmjs.org/install.sh | sudo sh
\end{minted}

\question{5}{It is advised not to install npm as root. Why?}

\task{5}{Install Express and Mongoose}

\begin{minted}{bash}
sudo npm install express mongoose -g
\end{minted}

\question{5} It is advised to install Express globally, since it has a common executable. However, Mongoose should not be installed globally. Why?

\task{10}{Hello, World}

\begin{minted}{bash}
express HelloWorld
cd HelloWorld && npm install
npm install mongoose
echo node_modules > .gitignore
\end{minted}

\question{5} Why do you think we need to install Mongoose locally even though it is already installed globally?

\task{10}{Write Mongoose models}

\begin{enumerate*}
\item Import mongoose
\item Connect to the database
\item Create a status schema
\item Create a status model
\item Create a status instance, if none exist
\item Export the status model
\end{enumerate*}

In the file \verb!models.js!, type:

\begin{minted}{javascript}
/**
 * Module dependencies.
 */

var mongoose = require('mongoose');

mongoose.connect('mongodb://localhost/hello_world');

var StatusSchema = new mongoose.Schema({
    message: String
});

var Status = mongoose.model('Status', StatusSchema);

Status.count(function(err, c) {
    if (c == 0) {
        var status = new Status();
        status.message = 'Hello, World';
        status.save()
    }
});

exports.Status = Status;
\end{minted}

\task{10}{Write an initial route and view}

In the file \verb!routes/index.js!, import the Status model by adding the line:

\begin{minted}{javascript}
var Status = require('../models').Status;
\end{minted}

Replace the basic view rendering (the line beginning \verb!res.render!) with:

\begin{minted}{javascript}
Status.find(function(err, statuses) {
    res.render('index', { title: 'Express', statuses: statuses });
});
\end{minted}

In the file \verb!views/index.jade!, add:

\begin{minted}{jade}
h2 Statuses
each status in statuses
  p #{status.message}
\end{minted}

\subsection*{Run your app}

Run the app by typing at a shell:

\begin{minted}{bash}
node app.js
\end{minted}

Now access the URL \href{http://localhost:3000/}{http://localhost:3000}; you should see a status-list page.

\subsection*{Turn-in instructions}
Create a personal Git repository on GitHub named ``Lab\labnumber-username'', where \verb!username! is your Kerberos username, and commit the following files to it:
\begin{itemize}
\item Edited project files from \productname (if any)
\item A PDF file with the answers to each question \textit{clearly marked}. When you have completed this lab, you will have answered \total{questions} questions.
\item A \verb!who.txt! file with your name. If you worked with someone else, include their name in the \verb!who.txt! file.
\end{itemize}

Once committed, push all files to GitHub. You may work with a partner, as long as each of you pushes your own copy of all of the listed files. Note that files submitted via ANGEL, email, or other methods will receive zero credit. In addition, files that cannot be accessed without additional work (e.g. zip files, LaTeX documents that require compiling) will lose you points. After you push your files to GitHub, please complete the feedback survey on Angel Under Labs - Labs Anonymous Feedback!


\subsection*{Revision History}
\begin{itemize*}
 \item April 12, 2012: Lab converted to LaTeX by Tim Ekl
 \item March 9, 2012: Lab written by Samad Jawaid
\end{itemize*}

\end{document}
