%%%
% CSSE lab template
% Written March 28, 2011 by Tim Ekl
%
% This template is part of a template system; you should be generating a lab
% using the newlab.sh file. Once you have generated the lab, feel free to
% remove this comment.
%
% This lab has five TODO items that need to be completed manually after
% generating the lab from template. Suggestions are provided with each TODO.
%
% For more info and examples of labs built from this template, see:
% http://github.com/CSSE376/Labs (note you must be a 376 TA to access)
%%%

\documentclass{article}

\newcommand{\labduedate}{at the end of Week 2}

% Actual required things start here
\usepackage{array}
\usepackage{mdwlist}
\usepackage{fancyhdr}
\usepackage[usenames,dvipsnames]{color}
\usepackage{graphicx}
\usepackage{minted}
\usepackage{fullpage}
\usepackage[colorlinks=true]{hyperref}
\usepackage{textcomp}
\usepackage{totcount}

% Helpful shortcuts for the document itself
\newcommand{\productname}{Node.js}
\newcommand{\longproductname}{Node.js (with Express and Mongoose)}
\newcommand{\labnumber}{4}


\newtotcounter{questions}\setcounter{questions}{0}
\newcommand{\question}[1]{
\addtocounter{questions}{1}
\ifnum1=#1
\textbf{[Question \arabic{questions} \textrm{\textit{(#1 pt)}}] }
\else
\textbf{[Question \arabic{questions} \textrm{\textit{(#1 pts)}}] }
\fi
}

\newtotcounter{tasks}\setcounter{tasks}{0}
\newcommand{\task}[2]{
\addtocounter{tasks}{1}
\setcounter{subtasks}{0}
\ifnum0=#1
\subsection*{Task \arabic{tasks}: #2}
\else
\subsection*{Task \arabic{tasks}: #2 \textit{(#1 pts)}}
\fi
}

\newtotcounter{subtasks}\setcounter{subtasks}{0}
\newcommand{\subtask}[2]{
\addtocounter{subtasks}{1}
\ifnum0=#1
\subsubsection*{Subtask \arabic{tasks}.\arabic{subtasks}: #2}
\else
\subsubsection*{Task \arabic{tasks}.\arabic{subtasks}: #2 \textit{(#1 pts)}}
\fi
}

\pagestyle{fancy}
\headheight 24pt
\begin{document}

\chead{\textcolor{Gray}{CSSE491 -- Scalable Computing Lab Assignment}}
\headsep = 24pt

\begin{center}
{ \large
\textbf{Lab \labnumber: \longproductname} \\
\textbf{Modifying Data}
}
\end{center}

\subsection*{Objective}
This lab builds on lab 1 to introduce students to the process of modifying data in a Node.js application using REST principles.

\subsection*{Lab format}
As with lab 1, this lab is almost entirely guided; it walks students through the process of adding a route and corresponding view to the existing HelloWorld Node.js application.

\subsection*{Required materials}
Students need a laptop, an Internet connection, and the HelloWorld app created in lab 1.

\subsection*{Grading rubric}
\begin{tabular}{p{5.5in} r}
Tasks 1 and 2 are worth \textbf{10} points each & $2 \times 10 = 20$ \\ \hline
& \textbf{20} points total
\end{tabular}

\subsection*{Due date}
The lab is due \textcolor{red}{\textbf{\labduedate}}.

\subsection*{Turn-in instructions}
Create a personal Git repository on GitHub named ``Lab\labnumber-username'', where \verb!username! is your Kerberos username, and commit the following files to it:
\begin{itemize}
\item Edited project files from \productname
\item A PDF file with the answers to each question \textit{clearly marked}. When you have completed this lab, you will have answered \total{questions} questions.
\item A \verb!who.txt! file with your name. If you worked with someone else, include their name in the \verb!who.txt! file.
\end{itemize}

Once committed, push all files to GitHub. You may work with a partner, as long as each of you pushes your own copy of all of the listed files. Note that files submitted via ANGEL, email, or other methods will receive zero credit. In addition, files that cannot be accessed without additional work (e.g. zip files, LaTeX documents that require compiling) will lose you points. After you push your files to GitHub, please complete the feedback survey on Angel Under Labs - Labs Anonymous Feedback!


\task{10}{Add a Route}

Open the file \verb!app.js!. In the Routes section, add:

\begin{minted}{javascript}
app.post('/statuses', routes.create_status)
\end{minted}

This binds the \verb!/statuses! path in your webapp to a function called \verb!create_status! in the routes definition file. Next we need to implement that function. Open the file \verb!routes/index.js! and add at the end:

\begin{minted}{javascript}
/*
 * POST create status
 */

exports.create_status = function(req, res) {
    var status = new Status();
    status.message = req.body.message;
    status.save();
    res.redirect('back');
};
\end{minted}

\task{10}{Add a view}

Now we can add statuses to our application, but we'd have to hit the \verb!/statuses! URL manually; let's make this easier on our users by changing our views to allow status creation from entirely within the webapp. Open the file \verb!views/index.jade! and add at the end:

\begin{minted}{jade}
h2 Create Status
form(method="post", action="/statuses")
  div
    label Message:
    input(name="message")
  div
    input(type="submit", value="Create status")
\end{minted}

\subsection*{Test the App}

At a shell, run the webapp with:

\begin{minted}{bash}
node app.js
\end{minted}

Check to see that you can add a status by going to \href{http://localhost:3000/}{http://localhost:3000} and using the status-creation box you see there.

\subsection*{Turn-in instructions}
Create a personal Git repository on GitHub named ``Lab\labnumber-username'', where \verb!username! is your Kerberos username, and commit the following files to it:
\begin{itemize}
\item Edited project files from \productname
\item A PDF file with the answers to each question \textit{clearly marked}. When you have completed this lab, you will have answered \total{questions} questions.
\item A \verb!who.txt! file with your name. If you worked with someone else, include their name in the \verb!who.txt! file.
\end{itemize}

Once committed, push all files to GitHub. You may work with a partner, as long as each of you pushes your own copy of all of the listed files. Note that files submitted via ANGEL, email, or other methods will receive zero credit. In addition, files that cannot be accessed without additional work (e.g. zip files, LaTeX documents that require compiling) will lose you points. After you push your files to GitHub, please complete the feedback survey on Angel Under Labs - Labs Anonymous Feedback!


\subsection*{Revision History}
\begin{itemize*}
\item April 13, 2012: Lab converted to LaTeX by Tim Ekl
\item March 13, 2012: Lab written by Samad Jawaid
\end{itemize*}

\end{document}
