%%%
% CSSE lab template
% Written March 28, 2011 by Tim Ekl
%
% This template is part of a template system; you should be generating a lab
% using the newlab.sh file. Once you have generated the lab, feel free to
% remove this comment.
%
% This lab has five TODO items that need to be completed manually after
% generating the lab from template. Suggestions are provided with each TODO.
%
% For more info and examples of labs built from this template, see:
% http://github.com/CSSE376/Labs (note you must be a 376 TA to access)
%%%

\documentclass{article}

\newcommand{\labduedate}{at the end of week 4}

% Actual required things start here
\usepackage{array}
\usepackage{mdwlist}
\usepackage{fancyhdr}
\usepackage[usenames,dvipsnames]{color}
\usepackage{graphicx}
\usepackage{minted}
\usepackage{fullpage}
\usepackage[colorlinks=true]{hyperref}
\usepackage{textcomp}
\usepackage{totcount}

% Helpful shortcuts for the document itself
\newcommand{\productname}{Azure}
\newcommand{\longproductname}{Windows Azure}
\newcommand{\labnumber}{7}


\newtotcounter{questions}\setcounter{questions}{0}
\newcommand{\question}[1]{
\addtocounter{questions}{1}
\ifnum1=#1
\textbf{[Question \arabic{questions} \textrm{\textit{(#1 pt)}}] }
\else
\textbf{[Question \arabic{questions} \textrm{\textit{(#1 pts)}}] }
\fi
}

\newtotcounter{tasks}\setcounter{tasks}{0}
\newcommand{\task}[2]{
\addtocounter{tasks}{1}
\setcounter{subtasks}{0}
\ifnum0=#1
\subsection*{Task \arabic{tasks}: #2}
\else
\subsection*{Task \arabic{tasks}: #2 \textit{(#1 pts)}}
\fi
}

\newtotcounter{subtasks}\setcounter{subtasks}{0}
\newcommand{\subtask}[2]{
\addtocounter{subtasks}{1}
\ifnum0=#1
\subsubsection*{Subtask \arabic{tasks}.\arabic{subtasks}: #2}
\else
\subsubsection*{Task \arabic{tasks}.\arabic{subtasks}: #2 \textit{(#1 pts)}}
\fi
}

\pagestyle{fancy}
\headheight 24pt
\begin{document}

\chead{\textcolor{Gray}{CSSE491 -- Scalable Computing Lab Assignment}}
\headsep = 24pt

\begin{center}
{ \large
\textbf{Lab \labnumber: \longproductname} \\
\textbf{Photos}
}
\end{center}

\subsection*{Objective}
This lab continues to build on the HelloWorld example from previous labs, and is meant to demonstrate how Node.js apps can provide ``real-world'' functionality to web applications. By the end of this lab, students should have their HelloWorld app hosting photos.

\subsection*{Lab format}
Like previous labs, this lab primarily guides students through the process of adding photos, with comprehension questions placed throughout.

\subsection*{Required materials}
Students will need:
\begin{itemize*}
\item Their completed HelloWorld app from the end of Lab 3
\item A laptop with an internet connection
\item Photos from \href{http://interfacelift.com}{interfacelift} in \verb!public/img/interfacelift!
\end{itemize*}


\subsection*{Grading rubric}
\begin{tabular}{p{5.5in} r}
Task 1 is worth \textbf{5} points. & $1 \times 5 = 5$ \\
Tasks 2-4 are worth \textbf{10} points each. & $3 \times 10 = 30$ \\
Questions 1-5 are worth \textbf{5} points each. & $5 \times 5 = 25$ \\
Question 6 is worth \textbf{10} points. & $1 \times 10 = 10$ \\ \hline
& \textbf{70} points
\end{tabular}

\subsection*{Due date}
The lab is due \textcolor{red}{\textbf{\labduedate}}.

\subsection*{Turn-in instructions}
Create a personal Git repository on GitHub named ``Lab\labnumber-username'', where \verb!username! is your Kerberos username, and commit the following files to it:
\begin{itemize}
\item Edited project files from \productname (if any)
\item A PDF file with the answers to each question \textit{clearly marked}. When you have completed this lab, you will have answered \total{questions} questions.
\item A \verb!who.txt! file with your name. If you worked with someone else, include their name in the \verb!who.txt! file.
\end{itemize}

Once committed, push all files to GitHub. You may work with a partner, as long as each of you pushes your own copy of all of the listed files. Note that files submitted via ANGEL, email, or other methods will receive zero credit. In addition, files that cannot be accessed without additional work (e.g. zip files, LaTeX documents that require compiling) will lose you points. After you push your files to GitHub, please complete the feedback survey on Angel Under Labs - Labs Anonymous Feedback!


\task{5}{Update the Model}

As in all new data entities, we first need to modify the model to provide information about the Photo type. Open \verb!models.js! and add:

\begin{minted}{javascript}
var PhotoSchema = new mongoose.Schema({
    path : String // src
  , description : String // alt
  , date : Date // timestamp
});

var Photo = mongoose.model('Photo', PhotoSchema);

Photo.count(function(err, c) {
  if (c == 0) {
    var FOLDER = 'public/img/interfacelift';
    fs.readdir(FOLDER, function(err, files) {
      for (var i in files) {
        var file = files[i];
        var photo = new Photo();
        photo.path = FOLDER.replace(FOLDER.split('/')[0], '') + '/' + file;
        photo.description = file.split('.')[0];
        photo.date = new Date();
        photo.save();
      }
    });
  }
});

exports.Photo = Photo;
\end{minted}

\question{5} When we store a Photo, what is stored in the database? What is stored on disk? Why do we make this tradeoff?

\question{5} Explain, in your own words, what the block beginning \verb!Photo.count...! does.

\task{10}{Adding Routes}

Now that our database understands photos, we need to expose an interface to users of our website to view those photos. Add to \verb!app.js!:

\begin{minted}{javascript}
routes.photos = require('./routes/photos.js');

app.get('/photos/:page', routes.photos.all);
\end{minted}

Now, open \verb!routes/photos.js! and add the following route:

\begin{minted}{javascript}
var Photo = require('../models').Photo;

exports.all = function(req, res) {
  var page = parseInt(req.params.page) || 0; // :page or all (first page)
  var options;
  if (page == 0) { // first page is a special case
    options = {limit: 2};
  } else {
    options = {skip: page - 1, limit: 3};
  }
  Photo.find({}, ['path', 'description'], options, function(err, photos) {
    var n = photos.length;
    var context;
    if (n == 0) {
      context = {'left': null, 'center': null, 'right': null};
    } else if (n == 1) {
      context = {'left': null, 'center': photos[0], 'right': null};
    } else if (n == 2) {
      if (page == 0) { // first page is a special case
        context = {'left': null, 'center': photos[0], 'right': photos[1]};
      } else {
        context = {'left': photos[0], 'center': photos[1], 'right': null};
      }
    } else if (n == 3) {
      context = {'left': photos[0], 'center': photos[1], 'right': photos[2]};
    }
    context.page = page;
    context.title = 'Photos';
    context.stylesheet = 'photos.css';
    res.render('photos', context);
  });
};
\end{minted}

\question{5} Can you think of a way to shorten the large \verb!if! block in the middle of this route?

\question{5} What will the value of the variable \verb!page! be if a user accesses the route \verb!/photos/all!? \verb!/photos/2!? \verb!/photos/null!?

\task{10}{Adding a View}

You'll notice that the end of the route from the previous task attempts to render a nonexistent page. Let's go ahead and create that page now. Open the file \verb!views/photos.jade! and insert:

\begin{minted}{jade}
if center
  .row
    .span2
      if left
        a(href='/photos/' + (page - 1))
          img(src=left.path, alt=left.description)
      else
        &nbsp;
    .span6.offset1
      a(href='/photo/' + center.description)
        img(src=center.path, alt=center.description)
    .span2.offset1
      if right
        a(href='/photos/' + (page + 1))
          img(src=right.path, alt=right.description)
      else
        &nbsp;
  br
div.alert.alert-info="All photos licensed by InterfaceLIFT.com for personal use only."
\end{minted}

\question{5} Does this view correctly handle edge cases for values of the \verb!page! variable? If not, how does it break?

\subsection*{Running the App}

As always, run the app (and check new functionality) by running:

\verb!node app.js!

Then access the app at \href{http://localhost:3000/photos}{http://localhost:3000/photos}.

\task{10}{Refactoring}

If you identified any issues in the process of answering the questions above, fix them in the app. Be sure to \textbf{commit your changes separately} from any code from other tasks.

\question{10} What did you change? Why?

\subsection*{Turn-in instructions}
Create a personal Git repository on GitHub named ``Lab\labnumber-username'', where \verb!username! is your Kerberos username, and commit the following files to it:
\begin{itemize}
\item Edited project files from \productname (if any)
\item A PDF file with the answers to each question \textit{clearly marked}. When you have completed this lab, you will have answered \total{questions} questions.
\item A \verb!who.txt! file with your name. If you worked with someone else, include their name in the \verb!who.txt! file.
\end{itemize}

Once committed, push all files to GitHub. You may work with a partner, as long as each of you pushes your own copy of all of the listed files. Note that files submitted via ANGEL, email, or other methods will receive zero credit. In addition, files that cannot be accessed without additional work (e.g. zip files, LaTeX documents that require compiling) will lose you points. After you push your files to GitHub, please complete the feedback survey on Angel Under Labs - Labs Anonymous Feedback!


\subsection*{Revision History}
\begin{itemize*}
\item April 14, 2012: Lab written by Samad Jawaid
\item April 16, 2012: Lab questions and instructions filled out by Tim Ekl
\end{itemize*}

\end{document}
